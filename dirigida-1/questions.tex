\begin{frame}
    \begin{enumerate}
        \item

              Considere las siguientes cantidades representadas en
              complemento a dos:

              \begin{equation*}
                  A=01100001,\quad
                  B=10011001.
              \end{equation*}

              Efectúe las siguientes operaciones, expresando el
              resultado en complemento a dos:

              \begin{multicols}{4}
                  \begin{enumerate}[a)]
                      \item

                            $A+B$.

                      \item

                            $A-B$.

                      \item

                            $B-A$.

                      \item

                            $-A-B$.
                  \end{enumerate}
              \end{multicols}

        \item

              Sean las cantidades expresadas en el sistema decimal
              de numeración $A=5$, $B=-7.5$, $C=3.25$ y $D=-30$.
              Se pide que

              \begin{enumerate}[a)]
                  \item

                        Represente $A$, $B$, $C$ y $D$ en complemento
                        a dos de $9$ bits.
                        Determine $A+D$, $A-D$ y $-A-D$ en dicho
                        formato así como el rango del mismo.

                  \item\label{q:b}

                        Represente $A$, $B$, $C$ y $D$ en coma
                        flotante con un $1$ bit para el signo, $4$
                        bits para el exponente y $4$ bits para la
                        mantisa.

                  \item

                        Determine el rango del formato detallado
                        en~\eqref{q:b}.
              \end{enumerate}

        \item

              Determine el valor decimal, la suma y la diferencia de
              los números binarios $A=11100111$ y $B=10111111$,
              suponiendo que

              \begin{multicols}{2}
                  \begin{enumerate}[a)]
                      \item

                            Ambos están representados en magnitud y
                            signo.

                      \item

                            Ambos están representados en complemento
                            a dos.
                  \end{enumerate}
              \end{multicols}

        \item

              Conteste verdadero o falso cada proposición.
              Justifique adecuadamente.

              \begin{enumerate}[a)]
                  \item

                        Al aproximar $\pi=\frac{22}{7}$ el error
                        relativo es aproximadamente
                        $4.024\times 10^{-5}$.

                  \item

                        Asuma un computador que usa $10$ bits.
                        El primer bit es para el signo, los
                        siguientes $4$ bits son para el exponente
                        (incluye el signo) y el resto de bits son
                        para la mantisa, entonces la representación
                        del número $6$ en esta máquina es
                        $0001111000$.

                  \item

                        Al evaluar $4.85274\times 0.0124758$ usando
                        aritmética de $4$ dígitos con redondeo se
                        obtiene $0.06057$.
              \end{enumerate}

              \saveenum
    \end{enumerate}
\end{frame}


\begin{frame}
    \begin{enumerate}
        \resume

        \item

              Evalúe
              \begin{math}
                  f\left(x\right)=
                  x^{3}-
                  6.1x^{2}+
                  3.2x+
                  1.5
              \end{math}
              en $x=4.71$ con una aritmética de tres cifras (por
              redondeo y por truncamiento) usando cada uno de los
              siguientes métodos:

              \begin{enumerate}[a)]
                  \item

                        Calcule cada sumando del polinomio
                        \begin{math}
                            \left(
                            x^{3},
                            -6.1x^{2},
                            3.2x,
                            1.5
                            \right)
                        \end{math}.

                  \item

                        Usando la forma anidada
                        \begin{math}
                            f\left(x\right)=
                            \left(
                            \left(x-6.1\right)x+
                            3.2
                            \right)x
                            +1.5
                        \end{math}.

                  \item

                        En cada caso hallar los errores relativos y
                        absolutos.
                        ¿Qué método brinda mayor exactitud?
              \end{enumerate}

        \item

              Calcula la suma y la resta de los números
              $a=0.4523\times 10^{4}$ y $b=0.2115\times 10^{-3}$
              con una aritmética flotante con mantisa de cuatro
              dígitos decimales, es decir, una aritmética de cuatro
              dígitos de precisión.
              ¿Se produce alguna diferencia cancelativa?

              \saveenum
    \end{enumerate}
\end{frame}

\begin{frame}
    \begin{enumerate}
        \resume

        \item

              Utilizando aritmética
    \end{enumerate}
\end{frame}
\begin{frame}
  \begin{enumerate}
    \item

          Considere las siguientes cantidades representadas en
          complemento a dos.

          \begin{equation*}
            A=01100001,\quad
            B=10011001.
          \end{equation*}

          Efectúe las siguientes operaciones, expresando el resultado
          en complemento a dos.

          \begin{multicols}{4}
            \begin{enumerate}[a)]
              \item

                    $A+B$.

              \item

                    $A-B$.

              \item

                    $B-A$.

              \item

                    $-A-B$.
            \end{enumerate}
          \end{multicols}

    \item

          Sean las cantidades expresadas en el sistema decimal de
          numeración $A=5$, $B=-7.5$, $C=3.25$ y $D=-30$.
          Se pide que

          \begin{enumerate}[a)]
            \item

                  Represente $A$, $B$, $C$ y $D$ en complemento a dos
                  de $9$ bits.
                  Determine $A+D$, $A-D$ y $-A-D$ en dicho formato
                  así como el rango del mismo.

            \item\label{q2:b}

                  Represente $A$, $B$, $C$ y $D$ en coma flotante con
                  un $1$ bit para el signo, $4$ bits para el
                  exponente y $4$ bits para la mantisa.

            \item

                  Determine el rango del formato detallado
                  en~\eqref{q2:b}.
          \end{enumerate}

    \item

          Determine el valor decimal, la suma y la diferencia de los
          números binarios $A=11100111$ y $B=10111111$, suponiendo
          que

          \begin{multicols}{2}
            \begin{enumerate}[a)]
              \item

                    Ambos están representados en magnitud y signo.

              \item

                    Ambos están representados en complemento a dos.
            \end{enumerate}
          \end{multicols}

    \item

          Conteste verdadero o falso cada proposición.
          Justifique adecuadamente.

          \begin{enumerate}[a)]
            \item

                  Al aproximar $\pi=\frac{22}{7}$ el error
                  relativo es aproximadamente
                  $4.024\times 10^{-5}$.

            \item

                  Asuma un computador que usa $10$ bits.
                  El primer bit es para el signo, los siguientes $4$
                  bits son para el exponente (incluye el signo) y el
                  resto de bits son para la mantisa, entonces la
                  representación del número $6$ en esta máquina es
                  $0001111000$.

            \item

                  Al evaluar $4.85274\times 0.0124758$ usando
                  aritmética de $4$ dígitos con redondeo se obtiene
                  $0.06057$.
          \end{enumerate}

    \item

          Evalúe
          \begin{math}
            f\left(x\right)=
            x^{3}-
            6.1x^{2}+
            3.2x+
            1.5
          \end{math}
          en $x=4.71$ con una aritmética de tres cifras (por redondeo
          y por truncamiento) usando cada uno de los siguientes
          métodos:

          \begin{enumerate}[a)]
            \item

                  Calcule cada sumando del polinomio
                  \begin{math}
                    \left(
                    x^{3},
                    -6.1x^{2},
                    3.2x,
                    1.5
                    \right)
                  \end{math}.

            \item

                  Usando la forma anidada
                  \begin{math}
                    f\left(x\right)=
                    \left(
                    \left(x-6.1\right)x+
                    3.2
                    \right)x
                    +1.5
                  \end{math}.

            \item

                  En cada caso hallar los errores relativos y
                  absolutos.
                  ¿Qué método brinda mayor exactitud?
          \end{enumerate}

          \saveenum
  \end{enumerate}
\end{frame}


\begin{frame}
  \begin{enumerate}
    \resume

    \item

          Calcula la suma y la resta de los números
          $a=0.4523\times 10^{4}$ y $b=0.2115\times 10^{-3}$
          con una aritmética flotante con mantisa de cuatro
          dígitos decimales, es decir, una aritmética de cuatro
          dígitos de precisión.
          ¿Se produce alguna diferencia cancelativa?

    \item

          Utilizando aritmética de $7$ dígitos, redondeo y
          considerando: $a=1234.567$, $b=1.234567$ y $c=3.333333$.

          \begin{multicols}{2}
            \begin{enumerate}[a)]
              \item

                    Calcule $\left(a+b\right)c$ y $\left(ac+bc\right)$.

              \item

                    Compare los errores relativos.
            \end{enumerate}
          \end{multicols}

    \item

          Considere los valores $A=0.492$, $B=0.603$, $C=-0.494$,
          $D=-0.602$, $E=10^{-5}$ y se desea calcular
          $F=\frac{A+B+C+D}{E}$.
          Se les brinda a dos alumnos una calculadora para realizar
          el cálculo y se les informa que la máquina trabaja con $3$
          dígitos en la mantisa, con redondeo y opera en base $10$.
          Efectuaron ese cálculo de forma distinta el alumno $X$
          calculó $A+B$ y después $C+D$, sumó los valores y dividió
          por $E$, mientras que el alumno $Y$ calculó $A+C$ y después
          $B+D$, sumó los valores y dividió por $E$.
          Realice los cálculos hechos por los dos alumnos y comente
          sobre los resultados obtenidos.
          Observe que se usaron procesos matemáticos equivalentes.

    \item

          Un computador que usa redondeo y punto flotante con $10$
          bits posee la siguiente estructura:
          el primer bit guarda información sobre el signo, los $3$
          bits siguientes guardan información sobre el
          exponente (desplazado $3$ unidades) y los $6$ bits
          restantes guardan los dígitos de la mantisa (a partir del
          segundo porque el primero siempre es uno y con redondeo en
          el séptimo dígito si esto es necesario).
          Por ejemplo, el registro $1110001000$ representa al número
          ${\left(-1\right)}^{1}\times 0.1001000\times 2{6-3}$.
          ¿Cómo almacena este computador al número $9.123$?
          Calcule el error relativo que se comete al realizar tal
          representación.

    \item

          Las raíces de la ecuación $ax^{2}+bx+c=0$ vienen dadas por
          $x_{1}=\frac{-b+\sqrt{b^2-4ac}}{2a}$,
          $x_{2}=\frac{-b-\sqrt{b^2-4ac}}{2a}$.
          Si $a=1$, $b=-0.3001$ y $c=0.00006$, entonces las raíces
          exactas son $x_{1}=0.29989993$ y
          $x_{2}=2.000667\times{10}^{-4}$.
          Use el sistema $\mathbb{F}\left(4,10,-10,10\right)$ y
          redondeo para calcular $x^{\ast}_{2}$ que es una
          aproximación de $x_{2}$.
          Use $\sqrt{0.09002}=0.30003331481$.

    \item

          ¿Cómo se debe evaluar la función
          \begin{math}
            f\left(x\right)=
            x-
            \sqrt{x^2-\alpha}
          \end{math}
          para $\alpha\ll x$?

          \saveenum
  \end{enumerate}
\end{frame}

\begin{frame}
  \begin{enumerate}
    \resume

    \item

          Asuma que se necesita calcular
          \begin{math}
            A=
            \sqrt{
              10^{14}+\frac{2}{3}
            }-
            10^7
          \end{math}
          en un computador que usa aritmética de punto flotante con una
          exactitud de $15$ dígitos.

          \begin{enumerate}[a)]
            \item Explicar si esta fórmula producirá información sin
                  pérdida de dígitos significativos.
                  ¿Cuál es el valor?

            \item\label{q12:a}

                  Reescribir la fórmula en una forma alternativa de
                  modo que un cálculo más exacto sea posible.
                  Compare con lo obtenido en la parte~\ref{q12:a}).
          \end{enumerate}

    \item

          Suponga que $x$ e $y$ son vectores de punto flotante y el
          producto $f\left(x,y\right)=\sum_{i=1}^{n}x_{i}y_{i}$
          es calculado usando aritmética de punto flotante.
          Muestre que el error en el cálculo realizado es

          \begin{equation*}
            \left|
            \widetilde{f}\left(x,y\right)-
            f\left(x,y\right)-
            \right|\leq
            n\epsilon
            {\left|x\right|}^{T}
            \left|y\right|+
            O\left(\epsilon^2\right).
          \end{equation*}


    \item

          Evalúe la función

          \begin{equation*}
            y_{1}\left(x\right)=
            \frac{\log\left(1+x\right)}{x}
          \end{equation*}
          para $x\approx 0$ usando doble precisión.
          Grafique la función en el intervalo
          $\left[-10^{-15},10^{-15}\right]$.
          Repita el experimento usando
          \begin{equation*}
            y_{2}\left(x\right)=
            \begin{cases}
              \frac{\log\left(1+x\right)}{\left(1+x\right)-1},
               & \text{si }1+x\neq 1 \\
              1,
               & \text{si } 1+x=1
            \end{cases}
          \end{equation*}

    \item

          Calcule un valor aproximado del épsilon de la máquina
          usando el algoritmo $1$.

          \begin{algorithm}[H]
            $A\leftarrow 1.0$\;
            $B\leftarrow 1.0$\;
            $p\leftarrow 0$\;
            \Mientras{\normalfont $\left(\left(A+1\right)-A\right)-1=0$}{
              $A\leftarrow 2\ast A$\;
              $p\leftarrow p+1$\;
              \Mientras{\normalfont $\left(\left(A+B\right)-A\right)-B\neq0$}{
                $B\leftarrow B+1$\;
              }
            }
          \end{algorithm}

          \saveenum
  \end{enumerate}
\end{frame}

\begin{frame}
  \begin{enumerate}

    \resume

    \item

          Demuestre que $\frac{4}{5}$ no se puede representar de
          manera exacta como número de máquina.
          ¿Cuál será el número de máquina más cercano?
          ¿Cuál será el error de redondeo relativo que se produce
          cuando se almacena internamente este número?

    \item

          Muestre ejemplos de que es posible que
          \begin{math}
            \operatorname{fl}
            \left[
              \operatorname{fl}
              \left(xy\right)
              z
              \right]\neq
            \operatorname{fl}
            \left[
              x
              \operatorname{fl}
              \left(yz\right)
              \right]
          \end{math}
          para números de máquina $x$, $y$ y $z$.

    \item

          Demuestre que si $x$, $y$ son números de máquina de $32$ bits y
          \begin{math}
            \left|y\right|\ll
            \left|x\right|2^{-25}
          \end{math}.

          \saveenum
  \end{enumerate}
\end{frame}